\chapter{Deterministic Finite Automata}

Circuits are a great model for bounded input lengths. But what if we want to compute on some unbounded input? We know we can determine a finite answer to an infinite class of questions using an algorithm. Can we use this to create a model of computation for unbounded inputs?

Consider XOR:
\[
    \text{XOR}: \{0,1\}^* \rightarrow \{0,1\}
\]
\begin{equation}
    \text{XOR}(x) = 
    \begin{cases}
        1 & \text{if number of inputs equal to 1 is odd} \\
        0 & \text{otherwise}
    \end{cases}
\end{equation}

We can define an algorithm for XOR as follows:
\begin{itemize}
    \item def XOR(x):
    \begin{itemize}
        \item ans = 0
        \item for i in range(len(x)):
        \begin{itemize}
            \item ans = (ans + x[i]) \% 2
        \end{itemize}
        \item return ans
    \end{itemize}
\end{itemize}

This is an exmaple of aa "Single Pass Constant Memory Algorithm." We can create a diagram to represent it as follows:
\begin{center}
    \img{./img/XOR-fsm.png}
\end{center}

\begin{definition}
    Deterministic Finite Automaton (DFA)

    DFA with c states over \{0, 1\} is a pair $D \equiv (T, S)$, where $T: [c] \times \{0, 1\} \rightarrow [c]$ and $S \subseteq [c]$. T is known as the transition function and defines the inputs that cause a transition in state. For example, if the current state is 0 and the input bit is 1, the transition function might output 1 to indicate a change in state from 0 to 1. S is the acceptor. 1 is output if the final state is a state $\in$ S.
\end{definition}

\begin{example}
    
    Consider the function below:
    \begin{equation}
        D(x) =
        \begin{cases}
            1 & \text{if number of 1's in x is divisble by 3} \\
            0 & \text{else}
        \end{cases}
    \end{equation}

    A DFA for this function is as follows
    \begin{center}
        \img{./img/dfa-2.png}
    \end{center}

    The transition function T is shown below and S, the set of accepting states is \{0\}. If we wanted to modify the function be 1 if number of 1's mod 3 was 1, we can simply modify S to \{1\}. 

    \begin{center}
        \begin{tabular}{ |c|c|c| }
            inital & bit & state \\
            \hline
            0 & 0 & 0 \\
            1 & 0 & 1 \\
            2 & 0 & 2 \\
            0 & 1 & 1 \\
            1 & 1 & 2 \\
            2 & 1 & 0 \\
        \end{tabular}
    \end{center}

\end{example}

\begin{example}
    \begin{equation}
        D(x) = 
        \begin{cases}
            1 & \text{if x ends in} 1 \\
            0 & \text{else}
        \end{cases}
    \end{equation}

    The DFA for the above function is as follows:
    \begin{center}
        \img{./img/dfa-3.png}
    \end{center}
\end{example}

\begin{example}
    \begin{equation}
        D(x) = 
        \begin{cases}
            1 & \text{if x has alternating bits} 1 \\
            0 & \text{else}
        \end{cases}
    \end{equation}

    For example, f(01010) = f(10101) = 1 while f(01001) = 0.

    The DFA for the above function is as follows:
    \begin{center}
        \img{./img/dfa-4.png}
    \end{center}
\end{example}

\begin{example}
    Design a DFS that outputs 1 on strings with the same start and end bit. For example, f(011110) = 1 while f(0110101) = 0.

    The DFA for the above function is as follows:
    \begin{center}
        \img{./img/dfa-5.png}
    \end{center}
\end{example}

\subsection*{Anatomy of a DFA}
Every DFA has \emph{unbounded input length} and bounded
\begin{itemize}
    \item number of states C
    \item transition function T
    \item set of accepting states S
\end{itemize}

Some more important symbols representing DFA parameters:
\begin{itemize}
    \item set of states Q of cardinality C
    \item alphabet $\Sigma$ (we will use binary)
    \item starting state $s_0$ (we will use 0)
    \item the language $L_f$, where $f$ is the function computed by D and $L_f = \{x: f(x) = 1\}$
\end{itemize}

\hr

\subsection*{Composition of DFA's}
It may be useful to put together several DFA's and connect them with an operation. For example, consider $f_1$ computed by DFA $D_1$ and $f_2$ computed by DFA $D_2$. We may want to find $f = f_1(x) \land f_2(x)$. Can we compute f using a single DFA?

It's important to once again note that DFA's can \kw{only} be used to compute single pass, constant memory algorithms. Therefore, we must find a way to compute $f_1$ and $f_2$ within the same pass if we want to compute $f$.

This may be trivial at times, such as in the case where $f_1$ determines if there are an even number of 1's and $f_2$ determines if there are a multiple of 3 1's. In this case, we need only create a DFA that outputs 1 if there are a multiple of 6 1's. Let's try to generalize this beyond the special case.

\theoremproof{
    DFA's are closed under "ANDs". $f_1, f_2$ are computable by DFA's $\iff f_1 \land f_2$ computable by DFA's.

    Put another way, if $L_1, L_2$ are recognized by DFA's, then there is a DFA that recognizes $L_1 \cap L_2$.
}
{
    We have $D_1 = (T_1, S_1)$ that computes $f_1$ with $C_1$ states and and $D_2 = (T_2, S_2)$ that computes $f_2$ with $C_2$ states. We want to compute $f_1 \land f_2$. The idea is to run $D_1$ and $D_2$ in parallel.

    We will have $C_1 \times C_2$ states represented as (i, j) pairs, where $i \in \{0, 1, ..., C_1 - 1\}$ and $j \in \{0, 1, ..., C_2 - 1\}.$

    We will define $T: (C_1 \times C_2) \times \{0,1\} \rightarrow C_1 \times C_2$. In particular, $T((i, j), a) = (T_1(i, a), T_2(j, a)).$ In other words, we simply use the transition functions from each DFA to find the next ordered pair to transition to. In this way, we are computing each function in parallel.

    Finally, we define $S = \{(i, j): i \in S_1 \text{ and } j \in S_2\}$. This completes the construction. 
}

We can repeat the same idea for OR, with the only difference being how we define S. For OR, $S = \{(i, j): i \in S_1 \text{ or } j \in S_2\}$. NOT is trivial as well, as we simply need to take the complement of the existing DFA's accepting set. 

\hr 

In practice, DFA's are often used for recognizing patterns. To discuss this, we introduce a new definition.

\begin{definition}
    Concatenation

    f, g are functions from binary strings to binary. The concatenation of f, g $f \circ g = 1 \iff f(x_1) = 1 \text{ and } g(x_2) = 1$ where $x_1, x_2$ are consecutive substrings of x.
\end{definition}

Can we compute the concatenation of any two functions using a DFA?

    