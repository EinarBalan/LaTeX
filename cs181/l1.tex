\chapter{Computation and Representation}

\kw{Major Idea}: What can we compute? What does it mean to compute?

\vspace*{0.5cm}

There are two aspects of computing: 
\begin{enumerate}
    \item Data (Representations of Objects)
    \begin{itemize}
        \item We can represent numbers in many ways i.e. roman numerals or the place-value system we are familiar with today
        \item Some representations are easier to work with than others -- to convey the distance to the moon in Roman numerals would fill a small book
        \item Clearly choosing the right representation can have a \emph{dramatic} effect on computation
    \end{itemize}
    \item Algorithms (Operations on Data)
    \begin{itemize}
        \item In general, there is more than one way to accomplish the same task; some better than others
        \item To multiply, we can either perform repeated additions or the typical gradeschool multiplication algorithm
        \item Gradeschool multiplication is far more efficient as an $O(n^2)$ algorithm compared the to the exponential nature of repeated addition
        \item Choosing the right algorithm is also incredibly important to computation 
    \end{itemize}
\end{enumerate}
\kw{Takeaway}: It is important to utilize both a good data represntation and a good algorithm in all computing tasks.

\vspace*{0.5cm}

\subsubsection*{Representations}
\begin{itemize}
    \item In general we can represent many objects as a sequence of 0's and 1's. Anything from images, text, video, audio, databases, etc. can be encoded in binary.
    \item BIG IDEA: We can compose representations in order to represent any object i.e. if you can represent objects of type T, then you can also represent lists of that object
\end{itemize}

\begin{definition}
    Representation, where E is one-to-one
    \[
    E: O \rightarrow \{ 0,1 \}^*
    \] 
\end{definition}

\begin{example}
    Represent natural numbers $E: N \rightarrow \{ 0, 1\}^*$

    Couple options:
    \begin{itemize}
        \item Unary: E(n) = 0000...0 (n + 1 zeroes)
        \item Binary: Standard binary encoding defined as follows
        \begin{equation}
            NtoB(n) =
            \begin{cases}
                0 & \text{if } n = 0\\
                1 & \text{if } n = 1\\
                NtoB(\lfloor \frac{n}{2} \rfloor)\circ(n \% 2) & \text{else}
            \end{cases}
        \end{equation}
        \item We know an encoding E is valid iff there exists a decoding function s.t. $D: \{0, 1\}^* \rightarrow O$ and $D(E(x)) = x$
    \end{itemize}
\end{example}

\begin{example}
    $E: Z \rightarrow \{0,1\}^*$
    \begin{equation}
        ZtoB(n) =
        \begin{cases}
            0 \circ NtoB(n) & \text{if } n >= 0\\
            1 \circ NtoB(-n) & \text{else}\\
        \end{cases}
    \end{equation}
\end{example}

\begin{example}
    Represent rational numbers $E: Q \rightarrow \{0,1\}^*$
    \begin{itemize}
        \item We know we can represent any rational number with a fraction i.e. a pair of numbers
        \item So if we can find a way to encode two numbers in a one-to-one way, we can represent any rational number
        \item Suggestions
        \begin{enumerate}
            \item Unary(p) $\circ$ 1 Unary(q)
            \item Encode length of numerator
            \item Add padding to shorter number to match lengths
            \item Utilize new $\overline{ZtoB}: Z \rightarrow \{0,1\}^*$ where each bit is duplicated and a 01 is added to the end
            \begin{itemize}
                \item $QtoS(p/q) = \overline{ZtoB}(p) \circ \overline{ZtoB}(q)$
            \end{itemize}
        \end{enumerate}
    \end{itemize}
\end{example}

\vspace{.5cm}

\kw{Big Idea}: We can represent objects and lists of objects as compositions of representations.