
\documentclass[11pt]{article}
  
%%% LATEX COMMON Packages 

\usepackage[bookmarks,colorlinks,breaklinks]{hyperref}  % PDF hyperlinks, with coloured links
\hypersetup{linkcolor=blue,citecolor=blue,filecolor=blue,urlcolor=blue} % all blue links
\usepackage{amsmath,amsfonts,amsthm,amssymb,environ,xstring}
\usepackage{dsfont}
\usepackage{fullpage}
\usepackage{mdframed}
\newcommand{\ignore}[1]{}
\newcommand \kw[1]{\textbf{#1}}
\newenvironment{answer}{
\vspace{.5cm}
\kw{Answer}
}
{

}

\title{\bf{CS 181: Homework 1}}
\author{ Einar Balan}
\date{}
\begin{document}
\maketitle

\begin{enumerate}
\item Let $S, T$ be two finite sets such that there is a one-to-one mapping from $S$ to $T$ and a one-to-one mapping from $T$ to $S$. Show that $|S| = |T|$ (i.e., the two sets have the same number of elements). [1 point]

\begin{answer}
    Towards a contradiction, assume $|S| \ne |T|$. This means that there must be at least one more element in S than T or vice versa. 
    
    Define $F: S \rightarrow T$ and $G: T \rightarrow S$, which are both one to one. 

    Because F is one to one, $\forall x \neq x' \in S \implies F(x) \neq F(x') \in T$. That is, for all distinct elements in S there is a unique mapping in T for that element. Consider the case where $|S| < |T|$. By the pidgeonhole principle, we know that there is at least one distinct x, x' pair s.t. their mappings are the same. Thus F cannot be one to one and we have reached a contradiction. Therefore $|S| \text{ is not} > |T|$.

    We repeat the same logic for G to conclude that $|T| \text{ is not} > |S|$. So $|S| = |T|$ if F and G are one to one. 
    
    \qed
\end{answer}
\pagebreak

\item Exercise 2.4 [1 point]. (Note: The constant 1000 is there for slack and not  a specific one.)

\begin{answer}
We can encode the graph using adjacency lists. For the vertex set $[n]$, we can produce a human readable encoding where the ith index of the list corresponds to a list of nodes that the ith node shares an edge with. We can then encode each value in the list using the optimal prefix free encoding described in class w/ length $|E(n)| + 2log_2(|E(x)|) + 2$. After doing this for each value, we can pass the list through the same encoder and concatenate the output of every list together. This will be a valid encoding. Put more formally,
\[
    l_i = \{PFE(n) : \text{ if edge (i, n) exists } \forall n \in [n]\}
\]

\[
    E: G_n \rightarrow \{0,1\}^{cnlogn}
\]
\[
    E = PFE(l_0) \circ PFE(l_1) \circ ... \circ PFE(l_n)
\]

In order to decode:
\begin{itemize}
    \item keep reading until 01 reached
    \item chop the part that was read off and decode the numbers using standard PFE decoder for natural numbers to get a list of neighbors in ith index
    \item draw graph w/ edges to neighbors indicated by list
    \item repeat until all bits have been read
\end{itemize}

To count the number of bits:
\begin{itemize}
    \item for $n \in N, |E(n)| = log_2(n)$
    \item the PFE of each number will take $log_2(n) + 2log_2(log_2(n)) + 2$ bits (this is the length of the more efficient PFE discussed briefly in class)
    \item so each list will be $10(log_2(n) + 2log_2(log_2(n)) + 2)$ bits
    \item the PFE of a list will then be $10(log_2(n) + 2log_2(log_2(n)) + 2) + 2log_2(10(log_2(n) + 2log_2(log_2(n)) + 2)) + 2$ bits
    \item so in total we have $n(10(log_2(n) + 2log_2(log_2(n)) + 2) + 2log_2(10(log_2(n) + 2log_2(log_2(n)) + 2)) + 2)$ bits since we have n lists 
    \item this is less than $1000nlog_2n$ for sufficiently large n 
\end{itemize}

\qed
\end{answer}

\pagebreak

\item Prove that the set $\{AND,NOT\}$ is universal. [1 point] 

\begin{answer}
As shown in class, NAND is universal. To show $\{\text{AND},\text{NOT}\}$ is universal, we can show that any NAND circit can be implemented using only AND/NOT and vice versa.
\begin{itemize}
    \item to construct NAND in terms of AND and NOT, we can replace very NAND gate with an AND gate followed by a NOT gate
    \item to construct NOT using only NAND
    \begin{itemize}
        \item NOT(a) = NAND(a, a)
    \end{itemize}
    \item to construct AND using only NAND
    \begin{itemize}
        \item AND(a, b) = NOT(NAND(a, b))
        \item AND(a, b) = NAND(NAND(a, b), NAND(a, b))
    \end{itemize}
\end{itemize}
So $\{\text{AND},\text{NOT}\}$ is universal. 

\qed
\end{answer}
\pagebreak

\item Exercise 3.4. To be more precise, the problem is asking you to show that there is a function that {\bf cannot} be computed by a Boolean circuit that is only allowed to use AND/OR (so NOT gates not allowed). As a further hint, you can show that there is a function that takes two inputs and has one output that cannot be computed in such a way (no matter how many AND/OR gates you use). [1 point]

\begin{answer}
    First, we show that the circuit described, $C$, is monotone. We can easily show that the operations AND and OR are monotone as follows:

    \begin{center}
        \begin{tabular}{ |c|c| }
            AND & \\
            \hline
            00 & 0 \\
            01 & 0 \\
            10 & 0 \\
            11 & 1 \\
        \end{tabular} 
        \; \;
        \begin{tabular}{ |c|c| }
            OR & \\
            \hline
            00 & 0 \\
            01 & 1 \\
            10 & 1 \\
            11 & 1 \\
        \end{tabular}
    \end{center}

    Both of these functions satisfy the property that if any bit in the input is changed from a 0 to a 1, the output can never decrease in value. To be more concrete, for two binary strings that are bitwise $\le$ each other, $C(x) \le C(x')$.

    If we have any composition of AND and OR (both of which we have shown to be monotone), we will also have a monotone function that satisfies $x, x' \in \{0,1\}^n, x_i \le x_i' \text{ for every } i \in [n] \implies C(x) \le C(x')$. Consider two monotonic functions $f$ and $g$. We know that if $x \le x'$, then $f(x) \le f(x')$. This implies that $g(f(x)) \le g(f(x'))$ since the output values of the function are no different than the input values and can be treated functionally identically as inputs to other monotone functions.

    Therefore, $C$ must be monotone since it is a composition of two monotone functions. Now we will show that $C$ cannot compute a function, and is therefore the set \{AND, OR\} is not universal. Consider $f$:
    \begin{center}
        \begin{tabular}{ |c|c| }
            $f$ & \\
            \hline
            00 & 1 \\
            01 & 0 \\
            10 & 0 \\
            11 & 0 \\
        \end{tabular}
    \end{center}
    $f$ is not monotone bc $f(00) > f(01)$ (changing the second bit from 0 to 1 causes a decrease in value). It follows that $C$ must not be able to compute $f$ since $C$ can only compute monotone functions. Therefore, $C$ is not universal.

    \qed
\end{answer}

\end{enumerate}

\end{document}
