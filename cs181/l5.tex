\chapter{Non-Deterministic Finite Automata}

To begin discussion of NFA's, we will first discuss function concatenations. For convenience, the definition is repeated below:

\begin{definition}
    Concatenation

    f, g are functions from binary strings to binary. The concatenation of f, g $f \circ g = 1 \iff f(x_1) = 1 \text{ and } g(x_2) = 1$ where $x_1, x_2$ are consecutive substrings of x.
\end{definition}

\begin{example}
    Function Concatenation

    Consider $f_1$ and $f_2$. $f_1$ returns 1 for all x and $f_2$ returns 1 if x starts with 1 and has length exactly 4. Both of these are computable by DFA's as shown below (the constant one is pretty obvious so it has been omitted). 

    \begin{center}
        \img{./img/dfa-6.png}
    \end{center}

    Can we compute the concatenation of these two functions using a DFA? In fact, we can. Using just 16 states, we can use a DFA to compute it. To clarify what exactly we're computing, $f_1 \circ f_2(001010) = 1$ since $f_1(00) = 1$ and $f_2(1010) = 1$.
\end{example}

\begin{example}
    
    Consider a function $f_{reverse}(x)$, which simply returns the value of f when input with the reverse of x i.e. the bits are written right to left. Can we compute this with a DFA? 

    It seems to be exactly the opposite of what we can compute using a DFA. DFA's always implement one pass, constant memory algorithms and it would seem necessary to violate this in order to implement $f_reverse$. We will revisit this subject. 
\end{example}

\subsection*{Non-Deterministic Finite Automata}
\begin{center}
    \img{./img/nfa-1.png}
\end{center}

NFA's are similar to DFA's, with some small variations:
\begin{itemize}
    \item they can have mulitple outgoing edges with the same states
    \item some edges can be missing (by convention, this indicates they lead to dead states)
    \item some edges are labeled $\varepsilon$
\end{itemize}

\begin{example}
    \begin{center}
        \img{./img/nfa-2.png}
    \end{center}

    The above represents the branching diagram of the NFA. If any branch has a state within S after the last bit is read, then 1 is output.
\end{example}

\begin{definition}
    Non-Deterministic Finite Automata

    An NFA, N is defined by a transition function and a set of accepting states. So N = (T, S).

    \begin{gather}
        T: [C] \times \{0, 1, \varepsilon\} \rightarrow \text{Power}([c]) \\
        S \subseteq [c] \\
        Power([c]) = \{I: I \subseteq [c] \}
    \end{gather}

    In other words, the transition function is mapping the current state and the current bit input to a subset of all states in the NFA.
\end{definition}

\subsection*{Why NFA's}
One might wonder, what are NFA's good for? Well, we can use them to compute the concatenation from earlier where $f_1$ is a constant and $f_2$ is 1 if the first bit is 0 and the input length is exactly 4. Logically, the concatenation of these two outputs 1 if the 4th bit from the end is 1 and 0 otherwise. The NFA is pictured below:
\begin{center}
    \img{./img/nfa-3.png}
\end{center}

Important Takeaway: In general, if two functions are computable by DFA's, we can compute their concatentation using an NFA in which we simply add $varepsilon$ transitions pointing from accepting states of the first function to the start of the second.
\begin{center}
    \img{./img/nfa-4.png}
\end{center}

\subsection*{Kleen* Operation on Functions}
We already discussed Kleen* as a set operation, but as a function operation it is defined as f*(x) = 1 if x can be broken up into $x_0, x_1, ..., x_n$ such that $f(x_i) = 1$ for all i.

Building upon the takeaway from earlier, if f is computable by a DFA, then f* is computable by an NFA. We simply add a dummy state leading into the previous DFA's initial state and add epsilon transitions pointing from all accepting states to the initial state.
\begin{center}
    \img{./img/nfa-5.png}
\end{center}