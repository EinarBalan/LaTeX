\chapter{Sets and Probablity Theory}

\section{Probabilistic Sets}
A random event, E, has more than 1 possible outcome in the sample space S. S is the collection of all possible event outcomes. We know that E $\subset$ S.

\begin{example}
    Number in dice roll
    \[S = \{1, 2, 3, 4, 5, 6\}\]
    \[E_{odd} = \{1, 3, 5\}\]
    \[E_{>3} = \{4, 5, 6\}\]
\end{example}

\subsection*{Operations}
We can apply several operations to our sets.

\begin{enumerate}
    \item Union, denoted $E_1 \cup E_2$
    \item Intersection, denoted $E_1 \cap E_2$ or $E_1E_2$
\end{enumerate}

\vspace{.5cm}

Consider $E_{odd}$ and $E_{>3}$ above.

\[E_{odd} \cup E_{>3} = \{1, 3, 4, 5, 6\}\]
\[E_{odd} \cap E_{>3} = \{5\}\]

These operations are commutative, associate, and distributive. Intersection has precedence over union. 

\vspace{.5cm}

\subsection*{Special Events}
\begin{itemize}
    \item S is the event the spans the entire sample space
    \item $\varnothing$ is the null event, it has no outcomes
    \item if $E_1$ and $E_2$ are mutually exclusive, $E_1E_2 = \varnothing$
    \item if $E_1$ and $E_2$ are collectively exhaustive, $E_1 \cup E_2 = S$
    \item  $\overline{E_1} = S - E_1$, the complement\footnote[1]{Demorgan's Laws hold} of $E_1$  
\end{itemize}

\vspace{2cm}

\subsection*{Frequentist Probability (Natural Variation)}
The probability of occurrence of E is the relative frequency of observations of E ina large number of repeated experiments. Put more formally below,
\[
    P(E) = \lim_{N\rightarrow\infty}\frac{n}{N} \text{, where n = occurences of E in N observations in S}
\]


\subsection*{Bayesian Probability (Incomplete Knowledge)}
The probability of an event E represents analysts' degree of belief that E will occur.

\vspace{.5cm}

\begin{center}
\begin{tabular}{p{5cm}|p{5cm}}
    Frequentist Probability & Bayesian Probability \\
    \hline
    probability of expecting a ground shaking intensity of 1g in next 100 years & probability of finding water on new planet \\
    \addlinespace
    max wind speed in a year & probability that a buidling will collapse under ground shaking intensity of 1g \\
    \addlinespace
    live load on a building & election results \\
    \addlinespace
    *based on previous observations & *not based on previous observations \\
    \addlinespace
    *cannot be reduced through more measurement & *can be reduced if more observations/measurements applied
\end{tabular}
\end{center}

\section{Axioms}
\begin{enumerate}
    \item $0 \le P(E) \le 1$
    \item  $P(S) = 1$
    \item $P(A \cup B) = P(A) + P(B)$, s.t. $AB = \varnothing$
\end{enumerate}
* these axioms are consistent with Frequentist probability

We can derive several rules from these axioms.
\begin{enumerate}
    \item $P(\overline{E}) = 1 - P(E)$
    \item $P(\varnothing) = 0$
    \item $P(E_1 \cup E_2) = P(E_1) + P(E_2) - P(E_1E_2)$
    \begin{itemize}
        \item if $E_1$ and $E_2$ are mutually exclusive, then we double count their intersection when using the 3rd axiom; subtracting it leads to the correct value
        \item what if we have > 2 events? Inclusion/Exclusion rule 
        \item $P(E_1 \cup E_2 \cup ... \cup E_n) =$ \[
             \sum_{i=1}^{n}P(E_i) - \sum_{i=1}^{n}\sum_{j=1}^{i-1}P(E_iE_j) + \sum_{i=1}^{n}\sum_{j=1}^{i-1}\sum_{k=1}^{j-1}P(E_iE_jE_k) + ... + (-1)^{n-1}P(E_1E_2...E_n)
            \]
    \end{itemize}
\end{enumerate}

\pagebreak

\subsection*{Conditional Probability}
We may want to determine the probability of an event given another event is guaranteed to occur. This is denoted $P(E_1 | E_2)$, which is read as $E_1$ given $E_2$. It essentially redefines the sample space to be $E_2$.


\begin{equation} \label{eq:1.1}
    P(E_1 | E_2) = \begin{cases}
        \frac{P(E_1E_2)}{P(E_2)} & P(E_2) > 0 \\
        0 & P(E_2) = 0
    \end{cases}
\end{equation}

From this equation, it follows that
\[
    P(E_1E_2) = P(E_1 | E_2)P(E_2)
\]

This holds in general for n events.
\[
    P(E_1E_2E_3) = P(E_1 | E_2E_3)P(E_2E_3)
                 = P(E_1 | E_2E_3)P(E_ 2 | E_3)P(E_3)
\]

\begin{example}
    Applying conditions to operations

    \[
        P(E_1 \cup E_2 | E_3)  = P(E_1 | E_3) + P(E_2 | E_3) - P(E_1E_2 | E_3)
    \]

    \[
        P(E_1E_2 | E_3)  = P(E_1|E_2|E_3)P(E_2|E_3) \text{, which follows from \ref{eq:1.1}}
    \]
\end{example}

\vspace{.5cm}

\subsection*{Independence}
Two events are indpendent iff $P(E_1|E_2) = P(E_1)$

We have mutual independence if $P(E_1E_2 \ldots E_n) = P(E_1)P(E_2) \ldots P(E_n)$.

\vspace{.5cm}

\subsection*{Theorem of Total Probability}
Consider an event A and a set of of mutually exclusive and collectively exhaustive events $E_1, E_2, \ldots, E_3$.

\begin{equation} \label{eq:1.2}
    P(A) = \sum_{i=1}^{n}P(A|E_i)P(E_i)
\end{equation}

\vspace{.5cm}

\vspace{.5cm}

\subsection*{Bayes' Rule}
Consider an event A and a set of of mutually exclusive and collectively exhaustive events $E_1, E_2, \ldots, E_3$ in S.

\begin{align*}
    P(AE_j) = P(E_j|A)P(A) = P(A|E_j)P(E_j) \\
    P(E_j|A) = \frac{P(A|E_j)P(E_j)}{P(A)} \\
    P(E_j|A) = \frac{P(A|E_j)P(E_j)}{\sum_{i=1}^{n}P(A|E_i)P(E_i)}\\ \text{where equation \ref{eq:1.2} is used to subtitute P(A)}
\end{align*}