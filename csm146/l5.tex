\chapter{Non-Linear Classifiers}
What if is impossible to create a good linear separation of the data? In this case, we may want to look into models that allow for non-linear decision bounaries.

\begin{center}
    \img{./img/non-linear-db.png}
\end{center}

\section{Neural Networks}
Designed to mimic the brain, neural networks are capable of producing non-linear decision boundary models. Each neural network, is made up of many \kw{nodes} connected by \kw{links}. Each link as an associated weight and activation level. Every node has an input function (typically just a weighted sum of inputs), an activation function, and an output. The following diagram models a single node:

\begin{center}
    \img{./img/nn-node.png}
\end{center}

There are many possible activation functions including:
\begin{itemize}
    \item sigmoid $\sigma(x) = \frac{1}{1 + e^-x}$
    \item hyperbolic tangent $tanh(x) = \frac{e6x-e^-x}{e^x+e^-x}$
    \item step function
    \item ReLU, which returns 0 if x $\leq$ 0 or x if x > 0
\end{itemize}

The steps for a "feed forward" neural network are the following (where x is input layer and $\theta^{(i)}$ is the weight matrix for layer i):
\begin{itemize}
    \item $z^{(2)} = \theta^{(1)}x$
    \item $a^{(2)} = g(z^{(2)})$
    \item concatenate 1 to start of $a^{(2)}$ (to account for bias)
    \item and repeat for remaining layers
\end{itemize}

\begin{example}
    
    let $\theta^{(1)} = \begin{bmatrix}
        1 & 0 & 0 & 1 \\
        0 & -1 & 1 & 0 \\
        2 & 0 & 1 & 1 
    \end{bmatrix}$, $\theta^{(2)} = \begin{bmatrix}
        0 & 1 & 1 & 0 
    \end{bmatrix}$, step function as activation function, and $x = \begin{bmatrix}
        1 \\
        0 \\
        2 \\
        1
    \end{bmatrix}$

    What is the output with x as input?

    $z^{(2)} = \theta^{(1)}x = \begin{bmatrix}
        2 \\ 2 \\ 5
    \end{bmatrix} \implies a^{(2)} = \begin{bmatrix}
        1 \\ 1 \\ 1 \\ 1
    \end{bmatrix} \implies z^{(3)} = 2 \implies$ output = 1. 
\end{example}

\begin{example}
    \begin{center}
        \img{./img/rectangle-nn.png}
    \end{center}

    We want to classify all points within the rectangle as positive, and all others as negative. Show a feed forward neural network that accomplishes this. 

    First, we notice that the rectangle consists of four lines, which can be representedas the following set of inequalities
    \begin{equation*}
        \begin{cases}
            x_1 > 0 & \\
            5 - x_1 > 0 & \\
            x_2 > 0 & \\
            2 - x_2 > 0 & \\
        \end{cases}
    \end{equation*}

    From this we construct our first weight matrix (with first column as constant, second as $x_1$ coefficient, and third as $x_2$ coefficient)
    \[ \theta^{(1)} = \begin{bmatrix}
        0 & 1 & 0 \\
        5 & -1 & 0 \\
        0 & 0 & 1 \\
        2 & 0 & -1 \\
    \end{bmatrix} 
    \]

    We want all values in $a^{(2)}$ to be 1 (indicating the point falls satisfies all the equations), so we the set the bias to be -3.5. Therefore,
    \[
        \theta^{(2)} = \begin{bmatrix}
            -3.5 & 1 & 1 & 1 & 1
        \end{bmatrix}
    \]

    We can take this further and construct any arbitrary decision boundary by combining examples such as the one above in multiple layers. 
\end{example}

\section{Training Neural Networks}
We can use stochastic gradient descent, except our loss function is changed to 
\[
    J(\theta)   = -\sum_i^m[y_ilogh_\theta(x_i) + (1-y_i)log(1-h_\theta(x_i))]
\]
How do we compute $\nabla J(\theta)$? We need to use an algorithm called back propogation.

\subsection*{Back Propogation}
Backpropogation utilizes the chain rule to compute a gradient.

TODO: wtf was he talking about bro https://medium.com/@karpathy/yes-you-should-understand-backprop-
e2f06eab496b
